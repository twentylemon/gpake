We start this protocol by describing the Password Authenticated Key Exchange by Juggling (J-PAKE) method
of Hao and Ryan~\cite{HaRy2010}.  Let $G$ be a subgroup of $\mathbb{Z}_p^* = \mathbb{Z}_p \setminus \{0\}$ 
with prime order $q$, where the Decision Diffie-Hellman (DDH) problem is considered intractable.  Let $g\in G$
be a generator of the subgroup and $s \in [1,q-1]$ be the shared password between Alice and Bob, where 
$[a,b]$ is notation meaning the elements of $\mathbb{Z}^*$ between $a$ and $b$:  $[a,b] := \{a,a+1,\dots,b \}$.
\\

The protocol consists of a setup round followed by two rounds of communication (for details on zero knowledge
proofs see sub-Section~\ref{sec:ZKP} -- in our implementation detailed later the XXX ZKP is used \comment{Which? Schnorr?}):
 \begin{itemize}
 \item[\textbf{Setup}] Alice picks $x_1 \in [0,q-1]$ and $x_2 \in [1,q-1]$ randomly
 \item[] Bob picks $x_3 \in [0,q-1]$ and $x_4 \in [1,q-1]$ randomly
 \item[]
 \item[\textbf{Round 1}] Alice sends $g^{x_1},g^{x_2}$ and zero knowledge proofs of $x_1$ and $x_2$ to Bob
 \item[] Bob sends $g^{x_3},g^{x_4}$ and zero knowledge proofs of $x_3$ and $x_4$ to Alice
 \item[] [Alice and Bob verify the knowledge proofs and that $g^{x_2},g^{x_4} \neq 1$]
 \item[]
 \item[\textbf{Round 2}] Alice sends $A = g^{(x_1+x_3+x_4)x_2\cdot s}$ and a zero knowledge proof of $x_2 s$
 \item[] Bob sends $B = g^{(x_1+x_2+x_3)x_4\cdot s}$ and a zero knowledge proof of $x_4 s$
 \item[]
 \end{itemize}

 At this point, both are able to determine 
 \[ K = \underbrace{\left(B / g^{x_2x_4 s} \right)^{x_2}}_\text{Computable by Alice} = g^{(x_1+x_3)x_2x_4s} = 
 \underbrace{\left(A / g^{x_2x_4 s} \right)^{x_4}}_\text{Computable by Bob}. \]
 The shared session key is then taken to be $\kappa = H(K)$, where $H$ is a hash function. 
 \comment{Why do they hash here?}  Note that this scheme has implicit key confirmation: each of Alice and 
 Bob believe only the other can calculate the shared key; an explicit key confirmation can then be 
 performed if desired (which will increase the number of rounds of communication by at least one).
 \\ 
 We now show that the J-PAKE protocol satisfies the four properties required to be considered a secure PAKE,
in order to illustrate the properties and the methods by which they can be proven:
without loss of generality we may assume that Alice is honest, and let $x_a := x_1+x_3+x_4$.  

\begin{lemma} With high probability (approximately $2^{-160}$ for $q$ a 160-bit prime) the element $g_a := g^{x_a}$ is a
generator for the subgroup $G$, and Alice can verify this.
\end{lemma}
\begin{proof}
Since $|G|=q$ is prime, it is sufficient to prove that $g_a \neq 1$ (any non-identity element generates $G$) with high probability.  
As Alice verifies that $x_3$ and $x_4$ are known to Bob due to his zero knowledge proofs in round 1, and $x_1$ is 
chosen randomly by Alice, $x_a$ must be a random value from Bob's perspective.  In other words, $x_a \neq 0$
with high probability, even if Bob is an active adversary.  Alice can verify that $g_a$ is a generator for $g$ as she knows the 
value of $x_a$.
\end{proof}

An analogous proof shows that when Bob is honest the element $g_b := g^{x_1+x_2+x_3}$ is a generator of $G$ with high probability.  We first show resistance to an offline attack with an active adversary.

\begin{theorem}[Offline resistance to active attack] Under the DDH assumption, when $g_a$ is a generator of $G$ any attacker Oscar cannot distinguish
Alice's ciphertext from a random non-identity element in the subgroup $G$.
\end{theorem}
\begin{proof}
Suppose that Alice communicates with Oscar, who does not know the password.  After the protocol is run, Oscar knows
\[ g^{x_1}, g^{x_2}, A = g_a^{x_2 s}, \text{ and ZKPs of the exponents } x_1 \text{ and } x_2. \]
By definition, with high probability the zero knowledge proofs reveal only one bit of information: that Alice knows the values of the
exponents.  As argued in the proof of the last lemma, with high probability $g_a$ is a (random) generator of the group $G$.  Furthermore, 
as $x_2$ is chosen randomly it follows that $x_2s \in [1,q-1]$ is random and thus unknown to Oscar.  Thus, the only way to distinguish $A$ from a random 
non-identity element would be for Oscar to solve an instance of the Decision Diffie-Hellman problem.
\end{proof}

The result in the case of a passive attacker follows in a straightforward manner (note that this case must be proven separately as above the active attacker does not know the password $s$, but when he passively observes a session Alice is communicating with Bob, who does know $s$).

\begin{theorem}[Offline resistance to passive attack] Under the DDH assumption, given that $g_a$ and $g_b$ are generators of $G$, the ciphertexts 
 \[ A = g_a^{x_2s} \text{ and } B = g_b^{x_4s} \]
do not leak information for password verification.
\end{theorem}
\begin{proof}
Our above work shows that the value $A$ looks random to Bob, and analogously that $B$ looks random to Alice.  Thus, both must look random
to a passive adversary, who has less information about the protocol's secrets than either Alice or Bob.
\end{proof}

Next we show forward secrecy under the assumption that the Square Computational Diffie-Hellman problem is hard (this problem, which has been shown to be equivalent to the Computational Diffie-Hellman problem, asks one to compute $g^{a^2}$ given the value $g^a$ with $a$ some unknown value).  Since the zero-knowledge proofs imply that Alice and Bob know the values of $x_1$ and $x_3$, with high probability $x_1+x_3 \neq 0$ in $\mathbb{Z}_q$ -- which implies $K = g^{(x_1+x_3)x_2x_4s} \neq 1$ with high probability, even if one of Alice or Bob is an active adversary.

\begin{theorem}[Forward Secrecy]
Under the Square Computational Diffie-Hellman assumption, when $K \neq 1$, past session keys derived from the protocol remain incomputable even when $s$ is later disclosed.
\end{theorem}
\begin{proof}
Knowing $s$, the attacker wants to compute $\kappa = H(K)$ given
\[ \{g^{x_1},g^{x_2},g^{x_3},g^{x_4},g^{(x_1+x^3+x_4)x_2}, g^{(x_1+x_2+x_3)x_4}. \} \]
Suppose the attacker can compute $K$, and thus $g^{(x_1+x_3)x_2x_4}$ from the above information -- we show how he can act as an oracle to solve the
Square Computational Diffie-Hellman problem.  Let $x_5 = x_1+x_3$ mod $q$ (which is non-zero when $K \neq 1$).  Then ...
\end{proof}
\comment{To be continued}




