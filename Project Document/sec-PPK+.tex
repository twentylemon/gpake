In this section, we present the Group PAKE extension of the PPK protocol, the simpler of the PAK/PPK
suite, using the Fairy Ring Dance construction from~\cite{HaYiChSh15}. The PPK+ protocol is described as follows:

\emph{Round 1}: Every participant $P_i$ selects $x_i\in_R \mathbb{Z}_q$ and $y_i\in_R \mathbb{Z}_q$ and 
broadcasts $m_{ij} = g^{x_i}\cdot (H_1(i, j, \pi))^r$ for all $j \neq i$ as well as $g^{y_i}$ together with a 
zero-knowledge proof, denoted as ZKP\{$y_i$\}, for proving the knowledge of the exponent $y_i$.

We define $z_i = y_{i+1} - y_{i-1}$. Then everyone is able to compute $g^{z_i} = g^{y_{i+1}}/g^{y_{i-1}}$. 
Every participant $P_i$ also computes $\sigma_{ij} = \left(\frac{m_{ji}}{H_1(j,i,\pi)^r}\right)^{x_i} = g^{x_i x_j}$
and checks:
\begin{itemize}
\item $g^{z_i} \neq 1$ for $i = 1, \ldots, n$.
\item $m_j \neq 0$ for $j = 1,\ldots,n, j\neq i$.
\item the received ZKP\{$y_j$\} for $j = 1,\ldots, n, j\neq i$ are valid.
\end{itemize}

Similar to all GPAKE constructions, the ZKP{$y_i$} are standard Schnorr non-interactive zero knowledge proofs outlined in~\cite{HaYiChSh15}.

\emph{Round 2}: Every participant $P_i$ broadcasts $(g^{z_i})^{y_i}$ and a zero knowledge proof,
ZKP\{$\tilde{y_i}$\} for providing the equality of the discrete logarithm of $(g^{z_i})^{y_i}$ to the base
$g^{z_i}$ and the discrete logarithm $g^{y_i}$ to the base $g$. Everyone then computes the raw pairwise
keys $K_{ij}$ according to PPK, namely $K_{ij} = H_3(i,j,m_{ij}, m_{ji}, \sigma_{ij}, \pi)$, and the derived
authentication and confirmation keys, $\kappa^{\text{MAC}} = H(K_{ij}, \text{``MAC"})$,
$\kappa^{\text{KC}} = H(K_{ij}, \text{``KC"})$. Furthermore, let
$A_{ij} = g^{y_i}\mid\mid \text{ZKP}\{y_i\}\mid\mid K_{ij}\mid\mid \text{ZKP}\{\tilde{y_i}\}$,
$P_i$ broadcast $t_{ij}^{\text{MAC}}  =\text{HMAC}(\kappa^{\text{MAC}}_{ij}, A_{ij})$ and
$t_{ij}^{\text{KC}}  = \text{HMAC}(\kappa^{\text{KC}}_{ij}, \text{``KC''}\mid\mid i\mid\mid j \mid\mid m_i \mid\mid m_j)$ for each $j \neq i$.


When this round finishes, everyone checks:
\begin{itemize}
\item the received ZKP\{$\tilde{y_j}$\} for $j = 1,\ldots, n, j\neq i$ are valid.
\item the received key confirmation strings $t^{\text{KC}}_{ji}$ for $j = 1,\ldots, n, j\neq i$ are valid.
\item the received message authentication tags $t^{\text{MAC}}_{ji}$ for $j = 1,\ldots, n, j\neq i$ are valid.
\end{itemize}

Again, the zero knowledge proofs are standard Chaum-Pedersen ZKP used in~\cite{HaYiChSh15}.

At the end of the two rounds, the same formula is used for calculating the group key. \comment{Yi: Steve please put the formula in section 3.1 and I'll reference it here.}

The security of the GPAKE protocol follows directly from the security of the two party protocol PPK.

For simplicity of our demonstration, there are a few tweaks we made in our implementation of the
protocol. Firstly, the supposedly independent random hash functions $H_1$ and $H_3$ are implemented
as \textit{SHA-256} shifted by two different constants. Obviously it has security implications on the
protocol, but it will certainly not be used in real life implementations, nor does it have any impact on the
run time performance that we are measuring.

Secondly, it should be noted that the formula for calculating the raw pairwise keys $K_{ij}$ at the end of
round 1 is not symmetric between $i$ and $j$. A modification is therefore made such that when
calculating $K_{ij}$, we set $i < j$ without loss of generality. In this case both parties would be able to
calculate the same pairwise key.














