In this section we present the Group PAKE extension of the PPK protocol, the simpler of the PAK/PPK
suite. The PPK+ protocol, which has the same setup as the PPK protocol of Section~\ref{sec:PPK}, is described as follows:
\\

\begin{itemize}
    \item[\textbf{(Round 1)}] Every participant $P_i$ selects $x_i\in_R \mathbb{Z}_q$ and $y_i\in_R \mathbb{Z}_q$ and 
        broadcasts, for all $i \neq j$, the values
        \[ m_{ij} = g^{x_i}\cdot (H_1(i, j, \pi))^r, \qquad g^{y_i}, \qquad \text{ZKP}\{y_i\}.\]
    \item[]
    \item[] Define $z_i = y_{i+1} / y_{i-1}$ (with cyclic index $i$). Each $P_i$ is able to compute $g^{z_i} = g^{y_{i+1}} / g^{y_{i-1}}$ and 
    \[ \sigma_{ij} = \left(\frac{m_{ji}}{H_1(j,i,\pi)^r}\right)^{x_i} = g^{x_i x_j},\] 
    and check: 
    \begin{itemize}
            \item $g^{z_i} \neq 1 \mod p$ for $i = 1, \ldots, n$;
            \item $m_{ij} \neq 0$ for $j \neq i$;
            \item the $\text{ZKP}\{y_j\}$ for all $j \neq i$ are valid.
        \end{itemize}
    \item[]
    \item[\textbf{(Round 2)}] Every participant $P_i$ broadcasts $(g^{z_i})^{y_i}$ and a zero knowledge proof
        ZKP\{$\tilde{y_i}$\} proving the equality of the discrete logarithm of $(g^{z_i})^{y_i}$ to the base
        $g^{z_i}$ and the discrete logarithm $g^{y_i}$ to the base $g$. Each member computes the pairwise PPK keys
        \[ K_{ij} = H_3(i,j,m_{ij}, m_{ji}, \sigma_{ij}, \pi), \]
        and the authentication and confirmation keys
        \[\kappa^{\text{MAC}} = H(K_{ij}, \text{``MAC''})\qquad\qquad \kappa^{\text{KC}} = H(K_{ij}, \text{``KC''}).\]
        Each member additionally broadcasts 
        \begin{align*}
        t_{ij}^{MAC} &= HMAC(\kappa_{ij}^{MAC},  g^{y_i} || \text{ZKP}\{y_i\} || (g^{z_i})^{y_i} || \text{ZKP}\{\tilde{y_i}\}) \\
        t_{ij}^{KC} &= HMAC(\kappa_{ij}^{KC}, ``KC'' || i || j || E_{ij} || E_{ji}).
        \end{align*}
    \item[]
    \item[]  Finally, all members confirm:
    \begin{itemize}
            \item the received ZKP\{$\tilde{y_j}$\} for $j \neq i$ are valid;
            \item the received key confirmation strings $t^{\text{KC}}_{ji}$ for $j \neq i$ are valid;
            \item the received message authentication tags $t^{\text{MAC}}_{ji}$ for $j \neq i$ are valid.
        \end{itemize}
        and establish the group key via Equation~\eqref{eq:key}, according to the Burmester-Desmedt group key agreement protocol.
        \item[]
\end{itemize}

We now go over a few details of our implementation of the
protocol. First, the supposedly independent random hash functions $H_1$ and $H_3$ are both the
 \textit{SHA-256} hash function shifted by two different constants. As with the modification for Dragonfly this is not
secure, however our implementation is simply used for comparing latency in the different GPAKEs -- this would be fixed 
before using the code for any other purpose.  Additionally,
as the formula for calculating the raw pairwise keys $K_{ij}$ at the end of
Round 1 above is not symmetric between $i$ and $j$, we calculate $K_{ij}$ where $i < j$ (practically, the 
participants can be ordered in some arbitrary way, such as lexicographically by their unique identifiers 
or by the time they accepted to enter the group protocol), then both parties $P_i$ and $P_j$ still calculate the same pairwise key $K_{i,j}$.


%\emph{Round 1}: Every participant $P_i$ selects $x_i\in_R \mathbb{Z}_q$ and $y_i\in_R \mathbb{Z}_q$ and 
%broadcasts $m_{ij} = g^{x_i}\cdot (H_1(i, j, \pi))^r$ for all $j \neq i$ as well as $g^{y_i}$ together with a 
%zero-knowledge proof, denoted as ZKP\{$y_i$\}, for proving the knowledge of the exponent $y_i$.
%
%We define $z_i = y_{i+1} - y_{i-1}$. Then everyone is able to compute $g^{z_i} = g^{y_{i+1}}/g^{y_{i-1}}$. 
%Every participant $P_i$ also computes $\sigma_{ij} = \left(\frac{m_{ji}}{H_1(j,i,\pi)^r}\right)^{x_i} = g^{x_i x_j}$
%and checks:
%\begin{itemize}
%\item $g^{z_i} \neq 1$ for $i = 1, \ldots, n$.
%\item $m_j \neq 0$ for $j = 1,\ldots,n, j\neq i$.
%\item the received ZKP\{$y_j$\} for $j = 1,\ldots, n, j\neq i$ are valid.
%\end{itemize}
%
%Similar to all GPAKE constructions, the ZKP{$y_i$} are standard Schnorr non-interactive zero knowledge proofs outlined in~\cite{HaYiChSh15}.
%
%\emph{Round 2}: Every participant $P_i$ broadcasts $(g^{z_i})^{y_i}$ and a zero knowledge proof,
%ZKP\{$\tilde{y_i}$\} for providing the equality of the discrete logarithm of $(g^{z_i})^{y_i}$ to the base
%$g^{z_i}$ and the discrete logarithm $g^{y_i}$ to the base $g$. Everyone then computes the raw pairwise
%keys $K_{ij}$ according to PPK, namely $K_{ij} = H_3(i,j,m_{ij}, m_{ji}, \sigma_{ij}, \pi)$, and the derived
%authentication and confirmation keys, $\kappa^{\text{MAC}} = H(K_{ij}, \text{``MAC"})$,
%$\kappa^{\text{KC}} = H(K_{ij}, \text{``KC"})$. Furthermore, let
%$A_{ij} = g^{y_i}\mid\mid \text{ZKP}\{y_i\}\mid\mid K_{ij}\mid\mid \text{ZKP}\{\tilde{y_i}\}$,
%$P_i$ broadcast $t_{ij}^{\text{MAC}}  =\text{HMAC}(\kappa^{\text{MAC}}_{ij}, A_{ij})$ and
%$t_{ij}^{\text{KC}}  = \text{HMAC}(\kappa^{\text{KC}}_{ij}, \text{``KC''}\mid\mid i\mid\mid j \mid\mid m_i \mid\mid m_j)$ for each $j \neq i$.
%
%
%When this round finishes, everyone checks:
%\begin{itemize}
%\item the received ZKP\{$\tilde{y_j}$\} for $j = 1,\ldots, n, j\neq i$ are valid.
%\item the received key confirmation strings $t^{\text{KC}}_{ji}$ for $j = 1,\ldots, n, j\neq i$ are valid.
%\item the received message authentication tags $t^{\text{MAC}}_{ji}$ for $j = 1,\ldots, n, j\neq i$ are valid.
%\end{itemize}
%
%Again, the zero knowledge proofs are standard Chaum-Pedersen ZKP used in~\cite{HaYiChSh15}.
%
%At the end of the two rounds, the same formula is used for calculating the group key. \comment{Yi: Steve please put the formula in section 3.1 and I'll reference it here.}
%














