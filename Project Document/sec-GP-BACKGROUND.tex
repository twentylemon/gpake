We now turn out attention to group setting, where $n$ agents all share knowledge of a common password $\pi$ and wish to establish a shared secure group key $K$.  Although there has been some previous work on this topic, much of it -- see, for intance, Dutta and Barua~\cite{DuBa06} -- required $O(n)$ rounds of communication to establish the shared group key, which is an issue as the protocol can easily be distrupted by one slow participant\footnote{To be fair, as mentioned below, the construction of Hao et. al. has security properties which are proven in an ad hoc way using some non-standard assumptions, such as the Algebraic Adversary Model.  Although the scheme of Dutta and Barua uses more rounds of communication, it is proven secure in the widely accepted BPR model of Bellare et. al.~\cite{BePoRo00} under the CDH assumption in the random oracle model.}.  In 2006, \comment{Steve to finish.}