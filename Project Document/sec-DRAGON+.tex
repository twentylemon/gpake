We now present our group extension of the Dragonfly protocol using the general construction.
The resulting GPAKE follows the Dragonfly protocol closely, with only minor modifications to the first round of communication
and the addition of a final round to establish the group key. The setup is the same as Dragonfly (see Section \ref{sec:Dragon}),
except an explicit generator $g$ of the subgroup $Q$ is required. The Dragonfly+ protocol is executed as follows:
\\

\begin{itemize}
    \item[\textbf{(Round 1)}] Every participant $P_i$ selects $r_{ij}, m_{ij} \in_R \mathbb{Z}_q^*$ for all $j \neq i$ and
        computes $s_{ij} = r_{ij} + m_{ij} \mod q$ along with the element 
        \[ E_{ij} = \pi^{-m_{ij}} \mod p.\] 
        If any $s_{ij} < 2$, $r_{ij}$ and $m_{ij}$
        must be re-established. Additionally, each member selects $y_i \in_R \mathbb{Z}_q$ and broadcasts the values
        \[ s_{ij},\qquad E_{ij}, \qquad g^{y_i} \text{ mod } p, \qquad \text{ZKP}\{y_i\}.\]
    \item[]
    \item[] Define $z_i = y_{i+1} / y_{i-1}$ (with cyclic index $i$). Each member is able to compute $g^{z_i} = g^{y_{i+1}} / g^{y_{i-1}}$, and check:
        \begin{itemize}
            \item $g^{z_i} \neq 1 \mod p$ for $i = 1, \ldots, n$;
            \item at least one of $E_{ij} \neq E_{ji}$ or $s_{ij} \neq s_{ji}$ is true for all $j \neq i$;
            \item the received $\text{ZKP}\{y_j\}$ for all $j \neq i$ is valid.
        \end{itemize}
    \item[]
    \item[\textbf{(Round 2)}] Every participant $P_i$ can compute the pairwise shared secrets $ss_{ij} = (\pi^{s_{ji}} E_{ji})^{r_{ij}} = \pi^{r_{ij} r_{ji}} \mod p$. They broadcast $A_{ij} = H(ss_{ij} || E_{ij} || s_{ij} || E_{ji} || s_{ji})$.
    \item[]
    \item[] Each participant confirms the hashes are correct.
    \item[]
    \item[\textbf{(Round 3)}] Every participant $P_i$ broadcasts $(g^{z_i})^{y_i}$ and a zero knowledge proof
        ZKP\{$\tilde{y_i}$\} proving the equality of the discrete logarithm of $(g^{z_i})^{y_i}$ to the base
        $g^{z_i}$ and the discrete logarithm $g^{y_i}$ to the base $g$. Each member computes the pairwise Dragonfly keys
        \[ K_{ij} = H(ss_{ij} || E_{ij} \times E_{ji} || (s_{ij} + s_{ji}) \mod q)\]
        and the authentication and confirmation keys
        \[\kappa^{\text{MAC}} = H(K_{ij}, \text{``MAC''})\qquad\qquad \kappa^{\text{KC}} = H(K_{ij}, \text{``KC''}).\]
        Each member additionally broadcasts 
         \begin{align*}
         t_{ij}^{MAC} &= HMAC(\kappa_{ij}^{MAC},  g^{y_i} || \text{ZKP}\{y_i\} || (g^{z_i})^{y_i} || \text{ZKP}\{\tilde{y_i}\}) \\
        t_{ij}^{KC} &= HMAC(\kappa_{ij}^{KC}, ``KC'' || i || j || E_{ij} || E_{ji}).
        \end{align*}
    \item[]
    \item[] Finally, all members confirm:
        \begin{itemize}
            \item the received ZKP\{$\tilde{y_j}$\} for $j \neq i$ are valid;
            \item the received key confirmation strings $t^{\text{KC}}_{ji}$ for $j \neq i$ are valid;
            \item the received message authentication tags $t^{\text{MAC}}_{ji}$ for $j \neq i$ are valid.
        \end{itemize}
        and establish the group key via Equation~\eqref{eq:key}, according to the Burmester-Desmedt group key agreement protocol. 
        \item[] 
\end{itemize}

Note that for simplicity in our implementation of Dragonfly+ we mapped the password directly to $g$, the generator of $Q$. This specific strategy is \emph{not} secure since the group element is not password dependent -- however, our implementation is simply used for comparing latency in the different GPAKEs, which will be relatively unaffected by the choice of generator the password is mapped to.  In a secure implementation the password should be mapped in a manner similar to Harkins~\cite{Ha15}.



