

\begin{itemize}
    \item implementation details (how password was mapped, source of primes, etc)
    \item change of small subgroup attack -- we don't restart the step, instead the other person verifies the value received
    \item brief overview of d+ -- can mostly reference dragonfly and general construction
    \item brief mention security
\end{itemize}

We present the group extension to the Dragonfly protocol using the general construction outlined in \cite{HaYiChSh15}.
The construction follows the Dragonfly protocol closely, with only minor modifications to the first round of communication
and of course adding the final round to establish the group key. The setup is the same as Dragonfly (see Section \ref{sec:Dragon}),
except the generator of the subgroup $Q$ is required, called $g$. The Dragonfly+ protocol executes as follows:

\emph{Preamble}: Every participant $P_i$ have a shared password which they map to an element $\pi \in Q$.

\emph{Round 1}: Every participant selects $r_{ij}, m_{ij} \in_R \mathbb{Z}_q^*$ for all $j \in \{1,\ldots,n\} \setminus \{i\}$.
They each compute $s_{ij} = r_{ij} + m_{ij} \mod q$ and the element $E_{ij} = \pi^{-m_{ij}} \mod p$. If any $s_{ij} < 2$, start this step over.
Each member then also selects $y_i \in_R \mathbb{Z}_q$. $P_i$ then broadcasts $s_{ij}, E_{ij}, g^{y_i} \mod p$ and $\text{ZKP}\{y_i\}$.
\comment{maybe just say each member performs a pairwise round 1 with each other member, and does $y_i$}

Define $z_i = y_{i+1} / y_{i-1}$ (with cyclic index $i$). Each member is able to compute $g^{z_i} = g^{y_{i+1}} / g^{y_{i-1}}$, and check:
\begin{itemize}
    \item $g^{z_i} \neq 1 \mod p$;
    \item One of $E_{ij} \neq E_{ji}$ or $s_{ij} \neq s_{ji}$ is true for all $j \in \{1,\ldots,n\} \setminus \{i\}$;
    \item the received $\text{ZKP}\{y_j\}$ for all $j \in \{1,\ldots,n\} \setminus \{i\}$ is valid.
\end{itemize}

\emph{Round 2}
\comment{compute shared secret, send hash}
\comment{verify hash}

\emph{Round 3}
\comment{compute pairwise key, group stuff}
